\documentclass{article}

\usepackage[table,xcdraw]{xcolor}
\usepackage{float}
\usepackage{fancyhdr}
\pagestyle{fancy}

\title{Parallel Orbital Simulation Environment (POSE)}
\author{Dylan R. Wagner}

\begin{document}
  \pagenumbering{gobble}
  \maketitle
  
  \section{Context}
  
  This document will describe the aspects of the simulation software including: operation, orbital model including models of perturbing forces, concurrent parallel design, input and output, model of orbital collisions and simulation of spacecraft logic. 
  
  \section{Summary of Operation}
  
  \paragraph{1)}
  
  The software is to take in parameters relating to the simulation environment. Read in the initial input file defining the state of objects in orbit at time 0. These objects will be added alongside entities pertaining to the orbital environment such as the Earth, Moon, Sun and other “Smart” objects. 
  
  \paragraph{2)}
  
  The simulation will then calculate acceleration vectors for each object in the environment then update the corresponding velocity vector and position. For “Smart” objects, custom logic evaluates the environment within the simulation then updates internal state. When the simulation has completed calculations for the current time interval, collision detection and modeling is evaluated. Any new bodies generated by the collision detection and modeling algorithm will be added to the current pool of bodies in the simulation. At this point, the current state of the bodies in the simulation will be logged to a file or sent to a remote location. 
  
  \paragraph{3)}
  
  From this point on, the simulation repeats until a stop condition is reached. This stop condition is tied to the total runtime or triggered by defined event(s) within the simulation. At the end of simulation statics will be logged to a flat file.
  
  \section{Orbital Simulation Method}
  
  The simulation software will use Cowell’s method. Cowell’s method involves adding together acceleration vectors acting on bodies in orbit. This summed acceleration vector can be separated in X, Y, Z components then integrated to find velocity with velocity being integrated to find spacial displacement. 
  
  \section{Models of Perturbing Forces}
  
  \begin{itemize}
  	\item Earth Gravity
  	\item Moon Gravity
  	\item Sun Gravity
  	\item Earth Non-spherical Gravity
  	\item Solar Radiation
  	\item Drag
  	\item Magnetic Fields
  	\item Propulsion
  \end{itemize}
  
  \section{Equations of Perturbing Forces (X, Y, Z)}
  
  \section{Concurrent Parallel Design}
  
  \section{Input and Output}
  
  \section{Model of Orbital Collisions}
  
  \section{Simulation of Spacecraft Logic}
  
\end{document}